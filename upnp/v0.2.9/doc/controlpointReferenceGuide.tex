\documentclass{article}
\usepackage{docxx}
\begin{document}
\pagebreak
\sloppy
\cxxTitle{}
        {EBS UPNP Control Point SDK}
        {}
        {}
        {}
\begin{cxxContents}
\cxxContentsEntry{1}{API Reference }{}
\cxxContentsEntry{2}{Internal Library Documentation}{}
\begin{cxxContents}
\cxxContentsEntry{2.1}{Discovery/SSDP}{}
\cxxContentsEntry{2.2}{Description}{}
\cxxContentsEntry{2.3}{Control/SOAP}{}
\cxxContentsEntry{2.4}{Eventing/GENA}{}
\end{cxxContents}
\end{cxxContents}
\clearpage\pagebreak
\begin{cxxdoc}
\<center\>
Embedded UPnP Library version 1.0 \<br\>
\<br\>
\</center\>
\end{cxxdoc}
\begin{cxxentry}
{}
        {API Reference }
        {}
        {}
        {1}
\begin{cxxnames}
\cxxitem{int}
        {UPnP\_RuntimeInit}
        {(\ UPnPRuntime*\ rt,\ \ UPNP\_UINT8*\ serverAddr,\ \ UPNP\_UINT16\ serverPort,\ \ UPNP\_INT16\ ipType,\ \ UPNP\_CHAR*\ wwwRootDir,\ \ UPNP\_INT16\ maxConnections,\ \ UPNP\_INT16\ maxHelperThreads\ )}
        {Initialize a UPnPRuntime }
        {1.1}
\cxxitem{void}
        {UPnP\_RuntimeDestroy}
        {(\ UPnPRuntime*\ rt\ )}
        {Destroy a UPnPRuntime }
        {1.2}
\cxxitem{int}
        {UPnP\_AddVirtualFile}
        {(\ UPnPRuntime*\ rt,\ const\ UPNP\_CHAR*\ serverPath,\ const\ UPNP\_UINT8*\ data,\ UPNP\_INT32\ size,\ const\ UPNP\_CHAR*\ contentType\ )}
        {Create a virtual file on the HTTP server. }
        {1.3}
\cxxitem{int}
        {UPnP\_RemoveVirtualFile}
        {(\ UPnPRuntime*\ rt,\ const\ UPNP\_CHAR*\ serverPath\ )}
        {Remove a virtual file from the server }
        {1.4}
\cxxitem{int}
        {UPnP\_ProcessState}
        {(\ UPnPRuntime*\ rt,\ \ UPNP\_INT32\ msecTimeout\ )}
        {Process asynchronous operations in non-threaded mode. }
        {1.5}
\cxxitem{int}
        {UPnP\_StartDaemon}
        {(\ UPnPRuntime*\ rt\ )}
        {Start the UPnP Daemon thread. }
        {1.6}
\cxxitem{int}
        {UPnP\_StopDaemon}
        {(\ UPnPRuntime*\ rt,\ UPNP\_INT32\ secTimeout\ )}
        {Kill the UPnP Daemon thread. }
        {1.7}
\cxxitem{const\ UPNP\_CHAR*}
        {UPnP\_GetPropertyValueByName}
        {(\ IXML\_Document*\ propertySet,\ \ const\ UPNP\_CHAR*\ name\ )}
        {Get the value of a named property in a GENA notify message. }
        {1.8}
\cxxitem{const\ UPNP\_CHAR*}
        {UPnP\_GetPropertyNameByIndex}
        {(\ IXML\_Document*\ propertySet,\ \ int\ index\ )}
        {Get the name of the nth property. }
        {1.9}
\cxxitem{const\ UPNP\_CHAR*}
        {UPnP\_GetPropertyValueByIndex}
        {(\ IXML\_Document*\ propertySet,\ \ int\ index\ )}
        {Get the value of the nth property. }
        {1.10}
\cxxitem{int}
        {UPnP\_AddToPropertySet}
        {(\ IXML\_Document**\ doc,\ \ const\ UPNP\_CHAR*\ name,\ \ const\ UPNP\_CHAR*\ value\ )}
        {Add name and value pair to GENA notify message property set. }
        {1.11}
\cxxitem{int}
        {UPnP\_CreateActionResponse}
        {(\ UPnPActionRequest*\ request\ \ )}
        {Creates a SOAP action response message. }
        {1.12}
\cxxitem{IXML\_Document*}
        {UPnP\_CreateAction}
        {(\ const\ UPNP\_CHAR*\ serviceTypeURI,\ \ const\ UPNP\_CHAR*\ actionName\ )}
        {Create a SOAP action request. }
        {1.13}
\cxxitem{int}
        {UPnP\_SetActionArg}
        {(\ IXML\_Document*\ actionDoc,\ \ const\ UPNP\_CHAR*\ name,\ \ const\ UPNP\_CHAR*\ value\ )}
        {Set an argument for a SOAP action response/request }
        {1.14}
\cxxitem{int}
        {UPnP\_ControlPointInit}
        {(\ UPnPControlPoint*\ cp,\ \ UPnPRuntime*\ rt,\ \ UPnPControlPointCallback\ callbackFn,\ \ void*\ callbackData\ )}
        {Initialize a UPnPControlPoint }
        {1.15}
\cxxitem{void}
        {UPnP\_ControlPointDestroy}
        {(\ UPnPControlPoint*\ cp,\ \ UPNP\_INT32\ gracefulTimeoutMsec\ )}
        {Destroy a UPnPControlPoint }
        {1.16}
\cxxitem{int}
        {UPnP\_ControlFindAll}
        {(\ UPnPControlPoint*\ cp,\ \ UPNP\_INT32\ timeoutMsec,\ \ void*\ userData,\ \ UPNP\_BOOL\ waitForCompletion\ )}
        {Search for UPnP devices. }
        {1.17}
\cxxitem{int}
        {UPnP\_ControlFindAllDevices}
        {(\ UPnPControlPoint*\ cp,\ \ UPNP\_INT32\ timeoutMsec,\ \ void*\ userData,\ \ UPNP\_BOOL\ waitForCompletion\ )}
        {Search for UPnP devices. }
        {1.18}
\cxxitem{int}
        {UPnP\_ControlFindDevicesByType}
        {(\ UPnPControlPoint*\ cp,\ \ UPNP\_CHAR*\ deviceType,\ \ UPNP\_INT32\ timeoutMsec,\ \ void*\ userData,\ \ UPNP\_BOOL\ waitForCompletion\ )}
        {Search for UPnP devices of a certain type. }
        {1.19}
\cxxitem{int}
        {UPnP\_ControlFindDevicesByUUID}
        {(\ UPnPControlPoint*\ cp,\ \ UPNP\_CHAR*\ uuid,\ \ UPNP\_INT32\ timeoutMsec,\ \ void*\ userData,\ \ UPNP\_BOOL\ waitForCompletion\ )}
        {Search for a particular UPnP device. }
        {1.20}
\cxxitem{int}
        {UPnP\_ControlFindDevicesByService}
        {(\ UPnPControlPoint*\ cp,\ \ UPNP\_CHAR*\ serviceType,\ \ UPNP\_INT32\ timeoutMsec,\ \ void*\ userData,\ \ UPNP\_BOOL\ waitForCompletion\ )}
        {Search for UPnP devices that offer a certain service. }
        {1.21}
\cxxitem{UPnPControlDeviceHandle}
        {UPnP\_ControlOpenDevice}
        {(\ UPnPControlPoint*\ cp,\ \ UPNP\_CHAR*\ url\ )}
        {Open a remote device for description, control, and eventing. }
        {1.22}
\cxxitem{int}
        {UPnP\_ControlOpenDeviceAsync}
        {(\ UPnPControlPoint*\ cp,\ \ UPNP\_CHAR*\ url,\ \ void*\ userData\ )}
        {Asynchronously open a remote device for description, control, and eventing. }
        {1.23}
\cxxitem{void}
        {UPnP\_ControlCloseDevice}
        {(\ UPnPControlDeviceHandle\ deviceHandle\ )}
        {Close an open device handle }
        {1.24}
\cxxitem{int}
        {UPnP\_ControlGetDeviceInfo}
        {(\ UPnPControlDeviceHandle\ device,\ \ UPnPControlDeviceInfo*\ info\ )}
        {Get information for an open device. }
        {1.25}
\cxxitem{int}
        {UPnP\_ControlGetServiceOwnerDeviceInfo}
        {(\ UPnPControlDeviceHandle\ handle,\ \ UPNP\_CHAR*\ serviceId,\ \ UPnPControlDeviceInfo*\ info\ )}
        {Get information for the parent device of a service }
        {1.26}
\cxxitem{UPNP\_CHAR*}
        {UPnP\_ControlGetServiceType}
        {(\ UPnPControlDeviceHandle\ deviceHandle,\ \ UPNP\_CHAR*\ serviceId\ )}
        {Get the type of a service }
        {1.27}
\cxxitem{int}
        {UPnP\_ControlGetServices}
        {(\ UPnPControlDeviceHandle\ deviceHandle,\ \ UPnPControlServiceIterator*\ i\ )}
        {Enumerate the services on a device }
        {1.28}
\cxxitem{int}
        {UPnP\_ControlGetServicesByType}
        {(\ UPnPControlDeviceHandle\ deviceHandle,\ \ UPnPControlServiceIterator*\ i,\ \ UPNP\_CHAR*\ serviceType\ )}
        {Enumerate the services of a certain type on a device }
        {1.29}
\cxxitem{UPNP\_CHAR*}
        {UPnP\_ControlNextService}
        {(\ UPnPControlServiceIterator*\ i\ )}
        {Enumerate the next service on the device. }
        {1.30}
\cxxitem{void}
        {UPnP\_ControlServiceIteratorDone}
        {(\ UPnPControlServiceIterator*\ i\ )}
        {Clean up when done enumerating services. }
        {1.31}
\cxxitem{IXML\_Document*}
        {UPnP\_ControlGetServiceInfo}
        {(\ UPnPControlDeviceHandle\ deviceHandle,\ \ UPNP\_CHAR*\ serviceId\ )}
        {Get detailed information about a service. }
        {1.32}
\cxxitem{int}
        {UPnP\_ControlGetServiceInfoAsync}
        {(\ UPnPControlDeviceHandle\ deviceHandle,\ \ UPNP\_CHAR*\ serviceId,\ \ void*\ userData\ )}
        {Asynchronous get detailed information about a service. }
        {1.33}
\cxxitem{int}
        {UPnP\_ControlInvokeAction}
        {(\ UPnPControlDeviceHandle\ deviceHandle,\ \ UPNP\_CHAR*\ serviceId,\ \ const\ UPNP\_CHAR*\ actionName,\ \ IXML\_Document*\ action,\ \ void*\ userData,\ \ UPNP\_BOOL\ waitForCompletion\ )}
        {Invoke an action on a remote service }
        {1.34}
\cxxitem{int}
        {UPnP\_ControlSubscribe}
        {(\ UPnPControlDeviceHandle\ deviceHandle,\ \ UPNP\_CHAR*\ serviceId,\ \ UPNP\_INT32\ timeoutSec,\ \ void*\ userData,\ \ UPNP\_BOOL\ waitForCompletion\ )}
        {Subscribe to a service or renew a subscription }
        {1.35}
\cxxitem{UPNP\_BOOL}
        {UPnP\_ControlSubscribedToService}
        {(\ UPnPControlDeviceHandle\ deviceHandle,\ \ UPNP\_CHAR*\ serviceId\ )}
        {Return whether or not the control point is subscribed to the given service }
        {1.36}
\cxxitem{void}
        {UPnP\_ControlSetServiceSubscriptionExpireWarning}
        {(\ UPnPControlDeviceHandle\ deviceHandle,\ \ UPNP\_CHAR*\ serviceId,\ \ UPNP\_INT32\ warningMsec\ )}
        {Sets the time offset before subscription expiration at which a warning event will be generated. }
        {1.37}
\cxxitem{int}
        {UPnP\_ControlUnsubscribe}
        {(\ UPnPControlDeviceHandle\ deviceHandle,\ \ UPNP\_CHAR*\ serviceId,\ \ void*\ userData,\ \ UPNP\_BOOL\ waitForCompletion\ )}
        {Cancel a subscribed service }
        {1.38}
\end{cxxnames}
\begin{cxxfunction}
{int}
        {UPnP\_RuntimeInit}
        {(\ UPnPRuntime*\ rt,\ \ UPNP\_UINT8*\ serverAddr,\ \ UPNP\_UINT16\ serverPort,\ \ UPNP\_INT16\ ipType,\ \ UPNP\_CHAR*\ wwwRootDir,\ \ UPNP\_INT16\ maxConnections,\ \ UPNP\_INT16\ maxHelperThreads\ )}
        {Initialize a UPnPRuntime }
        {1.1}
\cxxParameter{
\begin{tabular}[t]{lp{0.5\textwidth}}
{\tt\strut rt} & pointer to uninitialized UPnPRuntime struct 
\\
{\tt\strut serverAddr} & ip address to bind HTTP server to (NULL for IP\_ADDR\_ANY) 
\\
{\tt\strut serverPort} & port to bind HTTP server to (0 for ANY\_PORT) 
\\
{\tt\strut ipType} & type of ip version used (ipv4 or ipv6), (RTP\_NET\_TYPE\_IPV4 for v4 and RTP\_NET\_TYPE\_IPV6 for v6) 
\\
{\tt\strut wwwRootDir} & HTTP root dir on local file system 
\\
{\tt\strut maxConnections} & the maximum limit on simultaneous HTTP server connections 
\\
{\tt\strut maxHelperThreads} & if UPNP\_MULTITHREAD is defined, the max number of helper threads to spawn 
\end{tabular}}
\cxxReturn{
\begin{tabular}[t]{lp{0.5\textwidth}}
{\tt\strut error} & code
\end{tabular}}
\begin{cxxdoc}

Initializes the given UPnPRuntime struct, and sets up an HTTP
server instance to receive control/event messages.  This function
must be called before any other function in the UPnP SDK.


\end{cxxdoc}
\end{cxxfunction}
\begin{cxxfunction}
{void}
        {UPnP\_RuntimeDestroy}
        {(\ UPnPRuntime*\ rt\ )}
        {Destroy a UPnPRuntime }
        {1.2}
\cxxParameter{
\begin{tabular}[t]{lp{0.5\textwidth}}
{\tt\strut rt} & pointer to UPnPRuntime struct 
\end{tabular}}
\cxxReturn{
\begin{tabular}[t]{lp{0.5\textwidth}}
{\tt\strut error} & code
\end{tabular}}
\begin{cxxdoc}

Must be called after all other UPnP SDK calls to clean up runtime
data for UPnP.


\end{cxxdoc}
\end{cxxfunction}
\begin{cxxfunction}
{int}
        {UPnP\_AddVirtualFile}
        {(\ UPnPRuntime*\ rt,\ const\ UPNP\_CHAR*\ serverPath,\ const\ UPNP\_UINT8*\ data,\ UPNP\_INT32\ size,\ const\ UPNP\_CHAR*\ contentType\ )}
        {Create a virtual file on the HTTP server. }
        {1.3}
\cxxReturn{
\begin{tabular}[t]{lp{0.5\textwidth}}
{\tt\strut error} & code
\end{tabular}}
\cxxSee{UPnP\_RemoveVirtualFile
\strut}
\begin{cxxdoc}

Makes the data buffer passed in available at the given path on the
HTTP server.


\end{cxxdoc}
\end{cxxfunction}
\begin{cxxfunction}
{int}
        {UPnP\_RemoveVirtualFile}
        {(\ UPnPRuntime*\ rt,\ const\ UPNP\_CHAR*\ serverPath\ )}
        {Remove a virtual file from the server }
        {1.4}
\cxxReturn{
\begin{tabular}[t]{lp{0.5\textwidth}}
{\tt\strut error} & code
\end{tabular}}
\begin{cxxdoc}

Must be called before UPnP\_RuntimeDestroy to remove any virtual files
added using UPnP\_AddVirtualFile.


\end{cxxdoc}
\end{cxxfunction}
\begin{cxxfunction}
{int}
        {UPnP\_ProcessState}
        {(\ UPnPRuntime*\ rt,\ \ UPNP\_INT32\ msecTimeout\ )}
        {Process asynchronous operations in non-threaded mode. }
        {1.5}
\cxxParameter{
\begin{tabular}[t]{lp{0.5\textwidth}}
{\tt\strut rt} & pointer to UPnPRuntime struct 
\\
{\tt\strut msecTimeout} & time in miliseconds for which the function blocks 
\end{tabular}}
\cxxReturn{
\begin{tabular}[t]{lp{0.5\textwidth}}
{\tt\strut error} & code
\end{tabular}}
\begin{cxxdoc}

This function blocks for at most msecTimeout milliseconds, processing
any asynchronous operations that may be in progress on either the
control point or device runtime attached to the given UPnPRuntime.

This function must be called in order to receive events if an
application is running with the UPnP SDK in single-threaded mode.


\end{cxxdoc}
\end{cxxfunction}
\begin{cxxfunction}
{int}
        {UPnP\_StartDaemon}
        {(\ UPnPRuntime*\ rt\ )}
        {Start the UPnP Daemon thread. }
        {1.6}
\cxxParameter{
\begin{tabular}[t]{lp{0.5\textwidth}}
{\tt\strut rt} & pointer to UPnPRuntime struct 
\end{tabular}}
\cxxReturn{
\begin{tabular}[t]{lp{0.5\textwidth}}
{\tt\strut error} & code
\end{tabular}}
\cxxSee{UPnP\_StopDaemon
\strut}
\begin{cxxdoc}

This function must be called in multithreaded mode to start the
UPnP daemon, which listens for requests/announcements on the network,
and sends any events to the attached control point/device runtime.


\end{cxxdoc}
\end{cxxfunction}
\begin{cxxfunction}
{int}
        {UPnP\_StopDaemon}
        {(\ UPnPRuntime*\ rt,\ UPNP\_INT32\ secTimeout\ )}
        {Kill the UPnP Daemon thread. }
        {1.7}
\cxxParameter{
\begin{tabular}[t]{lp{0.5\textwidth}}
{\tt\strut rt} & the device runtime to stop 
\\
{\tt\strut secTimeout} & time to wait for daemon to stop 
\end{tabular}}
\cxxReturn{
\begin{tabular}[t]{lp{0.5\textwidth}}
{\tt\strut error} & code
\end{tabular}}
\begin{cxxdoc}

This function stops the UPnP daemon from executing.  It will wait
for at most secTimeout seconds for all helper threads to terminate.
If this function returns negative error code, it means the timeout
expired without the successful termination of one or more helper
threads.  In this case, calling UPnP\_RuntimeDestroy may cause a fault
since there are still helper threads running that may try to access
the data structures pointed to by the UPnPRuntime.


\end{cxxdoc}
\end{cxxfunction}
\begin{cxxfunction}
{const\ UPNP\_CHAR*}
        {UPnP\_GetPropertyValueByName}
        {(\ IXML\_Document*\ propertySet,\ \ const\ UPNP\_CHAR*\ name\ )}
        {Get the value of a named property in a GENA notify message. }
        {1.8}
\cxxReturn{
\begin{tabular}[t]{lp{0.5\textwidth}}
{\tt\strut the} & value or NULL if the property was not found
\end{tabular}}
\begin{cxxdoc}

The string returned must not be modified in any way.  It is valid until
the IXML\_Document is deleted.


\end{cxxdoc}
\end{cxxfunction}
\begin{cxxfunction}
{const\ UPNP\_CHAR*}
        {UPnP\_GetPropertyNameByIndex}
        {(\ IXML\_Document*\ propertySet,\ \ int\ index\ )}
        {Get the name of the nth property. }
        {1.9}
\cxxParameter{
\begin{tabular}[t]{lp{0.5\textwidth}}
{\tt\strut propertySet} & address of xml property set 
\\
{\tt\strut index} & index in property for value 
\end{tabular}}
\cxxReturn{
\begin{tabular}[t]{lp{0.5\textwidth}}
{\tt\strut the} & value or NULL if the property was not found
\end{tabular}}
\begin{cxxdoc}

The string returned must not be modified in any way.  It is valid until
the IXML\_Document is deleted.


\end{cxxdoc}
\end{cxxfunction}
\begin{cxxfunction}
{const\ UPNP\_CHAR*}
        {UPnP\_GetPropertyValueByIndex}
        {(\ IXML\_Document*\ propertySet,\ \ int\ index\ )}
        {Get the value of the nth property. }
        {1.10}
\cxxParameter{
\begin{tabular}[t]{lp{0.5\textwidth}}
{\tt\strut propertySet} & address of xml property set 
\\
{\tt\strut index} & index in property for value 
\end{tabular}}
\cxxReturn{
\begin{tabular}[t]{lp{0.5\textwidth}}
{\tt\strut the} & value or NULL if the property was not found
\end{tabular}}
\begin{cxxdoc}

The string returned must not be modified in any way.  It is valid until
the IXML\_Document is deleted.


\end{cxxdoc}
\end{cxxfunction}
\begin{cxxfunction}
{int}
        {UPnP\_AddToPropertySet}
        {(\ IXML\_Document**\ doc,\ \ const\ UPNP\_CHAR*\ name,\ \ const\ UPNP\_CHAR*\ value\ )}
        {Add name and value pair to GENA notify message property set. }
        {1.11}
\cxxParameter{
\begin{tabular}[t]{lp{0.5\textwidth}}
{\tt\strut doc} & address of property set 
\\
{\tt\strut name} & pointer to name for new entry 
\\
{\tt\strut value} & address of value of for the new entry 
\end{tabular}}
\cxxReturn{
\begin{tabular}[t]{lp{0.5\textwidth}}
{\tt\strut error} & code
\end{tabular}}
\begin{cxxdoc}

Add a new name value pair entry to the property set


\end{cxxdoc}
\end{cxxfunction}
\begin{cxxfunction}
{int}
        {UPnP\_CreateActionResponse}
        {(\ UPnPActionRequest*\ request\ \ )}
        {Creates a SOAP action response message. }
        {1.12}
\cxxReturn{
\begin{tabular}[t]{lp{0.5\textwidth}}
{\tt\strut error} & code
\end{tabular}}
\begin{cxxdoc}

Creates a response message skeleton for the supplied SOAP action request


\end{cxxdoc}
\end{cxxfunction}
\begin{cxxfunction}
{IXML\_Document*}
        {UPnP\_CreateAction}
        {(\ const\ UPNP\_CHAR*\ serviceTypeURI,\ \ const\ UPNP\_CHAR*\ actionName\ )}
        {Create a SOAP action request. }
        {1.13}
\cxxParameter{
\begin{tabular}[t]{lp{0.5\textwidth}}
{\tt\strut serviceTypeURI} & string containing service type of the target service 
\\
{\tt\strut actionName} & name on action on the target service 
\end{tabular}}
\cxxReturn{
\begin{tabular}[t]{lp{0.5\textwidth}}
{\tt\strut pointer} & to newly created IXML\_Document, which can be passed intoUPnP\_SetActionArg to set the action arguments; NULL on error
\end{tabular}}
\begin{cxxdoc}

Creates an XML document which will hold the SOAP action request
message. This function returns the address of newly formed XML document.
After finishing the process of sending action request the application
must release this xml document.


\end{cxxdoc}
\end{cxxfunction}
\begin{cxxfunction}
{int}
        {UPnP\_SetActionArg}
        {(\ IXML\_Document*\ actionDoc,\ \ const\ UPNP\_CHAR*\ name,\ \ const\ UPNP\_CHAR*\ value\ )}
        {Set an argument for a SOAP action response/request }
        {1.14}
\cxxParameter{
\begin{tabular}[t]{lp{0.5\textwidth}}
{\tt\strut actionDoc} & pointer to action respose message 
\\
{\tt\strut name} & argument name *
\\
{\tt\strut value} & argument value *
\end{tabular}}
\cxxReturn{
\begin{tabular}[t]{lp{0.5\textwidth}}
{\tt\strut error} & code
\end{tabular}}
\begin{cxxdoc}

This function can be used on an IXML\_Document created by either
UPnP\_CreateActionResponse ($\rightarrow$ 1.13, {\em page }\pageref{cxx.1.13}) or UPnP\_CreateAction to set
either the input or output arguments for a SOAP action.


\end{cxxdoc}
\end{cxxfunction}
\begin{cxxfunction}
{int}
        {UPnP\_ControlPointInit}
        {(\ UPnPControlPoint*\ cp,\ \ UPnPRuntime*\ rt,\ \ UPnPControlPointCallback\ callbackFn,\ \ void*\ callbackData\ )}
        {Initialize a UPnPControlPoint }
        {1.15}
\cxxParameter{
\begin{tabular}[t]{lp{0.5\textwidth}}
{\tt\strut cp} & pointer to control pointcontext instance 
\\
{\tt\strut rt} & UPnP runtime to associate withthis control point 
\\
{\tt\strut callbackFn} & callback for this controlpoint 
\\
{\tt\strut callbackData} & application-specific datawhich will be passed into thecallback 
\end{tabular}}
\cxxReturn{
\begin{tabular}[t]{lp{0.5\textwidth}}
{\tt\strut error} & code
\end{tabular}}
\cxxSee{UPnP\_ControlPointDestroy
\strut}
\begin{cxxdoc}

Initializes all control point state data in a UPnPControlPoint
struct (allocated by the calling application), and binds the
control point to the specified UPnPRuntime. The UPnPRuntime
must be initialized via UPnP\_RuntimeInit before this function
is called.  Only one control point may be bound to a single
UPnPRuntime at once.  This function must be called before
all other control point related functions.


\end{cxxdoc}
\end{cxxfunction}
\begin{cxxfunction}
{void}
        {UPnP\_ControlPointDestroy}
        {(\ UPnPControlPoint*\ cp,\ \ UPNP\_INT32\ gracefulTimeoutMsec\ )}
        {Destroy a UPnPControlPoint }
        {1.16}
\cxxParameter{
\begin{tabular}[t]{lp{0.5\textwidth}}
{\tt\strut cp} & the control point to destroy 
\\
{\tt\strut gracefulTimeoutMsec} & if this control point has anyoutstanding operations, wait forthis many milliseconds to allow themto complete gracefully.  Aftertimeout expires, do hard close. 
\end{tabular}}
\cxxReturn{
\begin{tabular}[t]{lp{0.5\textwidth}}
{\tt\strut error} & code
\end{tabular}}
\cxxSee{UPnP\_ControlPointInit
\strut}
\begin{cxxdoc}

Cleans up all data associated with a UPnPControlPoint structure.
Once this function has been called, it is safe to free the memory used
by the UPnPControlPoint structure.


\end{cxxdoc}
\end{cxxfunction}
\begin{cxxfunction}
{int}
        {UPnP\_ControlFindAll}
        {(\ UPnPControlPoint*\ cp,\ \ UPNP\_INT32\ timeoutMsec,\ \ void*\ userData,\ \ UPNP\_BOOL\ waitForCompletion\ )}
        {Search for UPnP devices. }
        {1.17}
\cxxParameter{
\begin{tabular}[t]{lp{0.5\textwidth}}
{\tt\strut cp} & control point, initialized byUPnP\_ControlPointInit 
\\
{\tt\strut timeoutMsec} & total duration of time, in milliseconds, for the search 
\\
{\tt\strut userData} & passed to callback as userRequestDatafor UPNP\_CONTROL\_EVENT\_DEVICE\_FOUNDevent 
\\
{\tt\strut waitForCompletion} & if UPNP\_TRUE, this function will notreturn until the search completes; elsethe function will return immediatelyafter sending the multicast searchrequest. 
\end{tabular}}
\cxxReturn{
\begin{tabular}[t]{lp{0.5\textwidth}}
{\tt\strut error} & code
\end{tabular}}
\cxxSee{UPnP\_ControlFindAllDevices
\strut}
\begin{cxxdoc}

Sends a request for all UPnP devices on the network to make their
presence known to the control point.  If this search method is
used, then seperate UPNP\_CONTROL\_EVENT\_DEVICE\_FOUND events will be
generated for each root device, embedded device, and service that
responds.

When the timeout has been reached, a UPNP\_CONTROL\_EVENT\_SEARCH\_COMPLETE

event will be sent to the control point.

Generates:
UPNP\_CONTROL\_EVENT\_DEVICE\_FOUND
UPNP\_CONTROL\_EVENT\_SEARCH\_COMPLETE


\end{cxxdoc}
\end{cxxfunction}
\begin{cxxfunction}
{int}
        {UPnP\_ControlFindAllDevices}
        {(\ UPnPControlPoint*\ cp,\ \ UPNP\_INT32\ timeoutMsec,\ \ void*\ userData,\ \ UPNP\_BOOL\ waitForCompletion\ )}
        {Search for UPnP devices. }
        {1.18}
\cxxParameter{
\begin{tabular}[t]{lp{0.5\textwidth}}
{\tt\strut cp} & control point, initialized byUPnP\_ControlPointInit 
\\
{\tt\strut timeoutMsec} & total duration of time, in milliseconds, for the search 
\\
{\tt\strut userData} & passed to callback as userRequestDatafor UPNP\_CONTROL\_EVENT\_DEVICE\_FOUNDevent 
\\
{\tt\strut waitForCompletion} & if UPNP\_TRUE, this function will notreturn until the search completes; elsethe function will return immediatelyafter sending the multicast searchrequest. 
\end{tabular}}
\cxxReturn{
\begin{tabular}[t]{lp{0.5\textwidth}}
{\tt\strut error} & code
\end{tabular}}
\cxxSee{UPnP\_ControlFindAll
\strut}
\begin{cxxdoc}

Sends a request for all UPnP devices on the network to make their
presence known to the control point.  Only one
UPNP\_CONTROL\_EVENT\_DEVICE\_FOUND event per root device will be generated
if this function is used.

When the timeout has been reached, a UPNP\_CONTROL\_EVENT\_SEARCH\_COMPLETE

event will be sent to the control point.

Generates:
UPNP\_CONTROL\_EVENT\_DEVICE\_FOUND
UPNP\_CONTROL\_EVENT\_SEARCH\_COMPLETE


\end{cxxdoc}
\end{cxxfunction}
\begin{cxxfunction}
{int}
        {UPnP\_ControlFindDevicesByType}
        {(\ UPnPControlPoint*\ cp,\ \ UPNP\_CHAR*\ deviceType,\ \ UPNP\_INT32\ timeoutMsec,\ \ void*\ userData,\ \ UPNP\_BOOL\ waitForCompletion\ )}
        {Search for UPnP devices of a certain type. }
        {1.19}
\cxxParameter{
\begin{tabular}[t]{lp{0.5\textwidth}}
{\tt\strut cp} & control point, initialized byUPnP\_ControlPointInit 
\\
{\tt\strut deviceType} & device type to search for, as specifiedby the UPnP Forum. 
\\
{\tt\strut timeoutMsec} & total duration of time, in milliseconds, for the search 
\\
{\tt\strut userData} & passed to callback as userRequestDatafor UPNP\_CONTROL\_EVENT\_DEVICE\_FOUNDevent 
\\
{\tt\strut waitForCompletion} & if UPNP\_TRUE, this function will notreturn until the search completes; elsethe function will return immediatelyafter sending the multicast searchrequest. 
\end{tabular}}
\cxxReturn{
\begin{tabular}[t]{lp{0.5\textwidth}}
{\tt\strut error} & code
\end{tabular}}
\cxxSee{UPnP\_ControlFindDevicesByUUID,\ UPnP\_ControlFindDevicesByService
\strut}
\begin{cxxdoc}

Sends a request for all UPnP devices of a certain type on the network
to make their presence known to the control point.  One
UPNP\_CONTROL\_EVENT\_DEVICE\_FOUND event per device that matches the
search type will be generated if this function is used.

When the timeout has been reached, a UPNP\_CONTROL\_EVENT\_SEARCH\_COMPLETE

event will be sent to the control point.

Generates:
UPNP\_CONTROL\_EVENT\_DEVICE\_FOUND
UPNP\_CONTROL\_EVENT\_SEARCH\_COMPLETE


\end{cxxdoc}
\end{cxxfunction}
\begin{cxxfunction}
{int}
        {UPnP\_ControlFindDevicesByUUID}
        {(\ UPnPControlPoint*\ cp,\ \ UPNP\_CHAR*\ uuid,\ \ UPNP\_INT32\ timeoutMsec,\ \ void*\ userData,\ \ UPNP\_BOOL\ waitForCompletion\ )}
        {Search for a particular UPnP device. }
        {1.20}
\cxxParameter{
\begin{tabular}[t]{lp{0.5\textwidth}}
{\tt\strut cp} & control point, initialized byUPnP\_ControlPointInit 
\\
{\tt\strut uuid} & uuid to search for, as specified by theUPnP Forum. 
\\
{\tt\strut timeoutMsec} & total duration of time, in milliseconds, for the search 
\\
{\tt\strut userData} & passed to callback as userRequestDatafor UPNP\_CONTROL\_EVENT\_DEVICE\_FOUNDevent 
\\
{\tt\strut waitForCompletion} & if UPNP\_TRUE, this function will notreturn until the search completes; elsethe function will return immediatelyafter sending the multicast searchrequest. 
\end{tabular}}
\cxxReturn{
\begin{tabular}[t]{lp{0.5\textwidth}}
{\tt\strut error} & code
\end{tabular}}
\cxxSee{UPnP\_ControlFindDevicesByType
\strut}
\begin{cxxdoc}

Sends a request for UPnP devices with the given UUID(unique identifier)
to make their presence known to the control point.  One
UPNP\_CONTROL\_EVENT\_DEVICE\_FOUND event per device that matches the
search type will be generated if this function is used.

When the timeout has been reached, a UPNP\_CONTROL\_EVENT\_SEARCH\_COMPLETE

event will be sent to the control point.

Generates:
UPNP\_CONTROL\_EVENT\_DEVICE\_FOUND
UPNP\_CONTROL\_EVENT\_SEARCH\_COMPLETE


\end{cxxdoc}
\end{cxxfunction}
\begin{cxxfunction}
{int}
        {UPnP\_ControlFindDevicesByService}
        {(\ UPnPControlPoint*\ cp,\ \ UPNP\_CHAR*\ serviceType,\ \ UPNP\_INT32\ timeoutMsec,\ \ void*\ userData,\ \ UPNP\_BOOL\ waitForCompletion\ )}
        {Search for UPnP devices that offer a certain service. }
        {1.21}
\cxxParameter{
\begin{tabular}[t]{lp{0.5\textwidth}}
{\tt\strut cp} & control point, initialized byUPnP\_ControlPointInit 
\\
{\tt\strut serviceType} & service type to search for, asspecified by the UPnP Forum. 
\\
{\tt\strut timeoutMsec} & total duration of time, in milliseconds,for the search 
\\
{\tt\strut userData} & passed to callback as userRequestDatafor UPNP\_CONTROL\_EVENT\_DEVICE\_FOUNDevent 
\\
{\tt\strut waitForCompletion} & if UPNP\_TRUE, this function will notreturn until the search completes; elsethe function will return immediatelyafter sending the multicast searchrequest. 
\end{tabular}}
\cxxReturn{
\begin{tabular}[t]{lp{0.5\textwidth}}
{\tt\strut error} & code
\end{tabular}}
\cxxSee{UPnP\_ControlFindDevicesByType
\strut}
\begin{cxxdoc}

Sends a request for UPnP devices with one or more services of the given
type to make their  presence known to the control point.  One
UPNP\_CONTROL\_EVENT\_DEVICE\_FOUND event per device that matches the
search type will be generated if this function is used.

When the timeout has been reached, a UPNP\_CONTROL\_EVENT\_SEARCH\_COMPLETE

event will be sent to the control point.

Generates:
UPNP\_CONTROL\_EVENT\_DEVICE\_FOUND
UPNP\_CONTROL\_EVENT\_SEARCH\_COMPLETE


\end{cxxdoc}
\end{cxxfunction}
\begin{cxxfunction}
{UPnPControlDeviceHandle}
        {UPnP\_ControlOpenDevice}
        {(\ UPnPControlPoint*\ cp,\ \ UPNP\_CHAR*\ url\ )}
        {Open a remote device for description, control, and eventing. }
        {1.22}
\cxxParameter{
\begin{tabular}[t]{lp{0.5\textwidth}}
{\tt\strut cp} & control point, initialized byUPnP\_ControlPointInit 
\\
{\tt\strut url} & url of device to open 
\end{tabular}}
\cxxReturn{
\begin{tabular}[t]{lp{0.5\textwidth}}
{\tt\strut UPnPControlDeviceHandle} & for the open device, or 0 on failure
\end{tabular}}
\cxxSee{UPnP\_ControlOpenDeviceAsync
\strut}
\begin{cxxdoc}

Opens the device at the specified URL.  This function is used to
obtain a UPnPControlDeviceHandle, which is used for most functions
which operate within the UPnP description, control, and eventing
phases.

This function will block until the device open has completed or failed.

Generates:
(no events)


\end{cxxdoc}
\end{cxxfunction}
\begin{cxxfunction}
{int}
        {UPnP\_ControlOpenDeviceAsync}
        {(\ UPnPControlPoint*\ cp,\ \ UPNP\_CHAR*\ url,\ \ void*\ userData\ )}
        {Asynchronously open a remote device for description, control, and eventing. }
        {1.23}
\cxxParameter{
\begin{tabular}[t]{lp{0.5\textwidth}}
{\tt\strut cp} & control point, initialized byUPnP\_ControlPointInit 
\\
{\tt\strut url} & url of device to open 
\\
{\tt\strut userData} & passed to callback as userRequestData forUPNP\_CONTROL\_EVENT\_DEVICE\_OPEN event 
\end{tabular}}
\cxxReturn{
\begin{tabular}[t]{lp{0.5\textwidth}}
{\tt\strut UPnPControlDeviceHandle} & for the open device, or 0 on failure
\end{tabular}}
\cxxSee{UPnP\_ControlOpenDeviceAsync
\strut}
\begin{cxxdoc}

Opens the device at the specified URL.  This function is used to
obtain a UPnPControlDeviceHandle, which is used for most functions
which operate within the UPnP description, control, and eventing
phases.

This function will return immediately if the control point is able to
initiate the device open successfully.  One of the events listed
below will be sent to the control point once the open completes (and
the UPnPControlDeviceHandle is available) or an error occurs.

Generates:
UPNP\_CONTROL\_EVENT\_DEVICE\_OPEN
UPNP\_CONTROL\_EVENT\_DEVICE\_OPEN\_FAILED


\end{cxxdoc}
\end{cxxfunction}
\begin{cxxfunction}
{void}
        {UPnP\_ControlCloseDevice}
        {(\ UPnPControlDeviceHandle\ deviceHandle\ )}
        {Close an open device handle }
        {1.24}
\cxxParameter{
\begin{tabular}[t]{lp{0.5\textwidth}}
{\tt\strut deviceHandle} & handle returned byUPnP\_ControlOpenDevice
\end{tabular}}
\cxxReturn{
\begin{tabular}[t]{lp{0.5\textwidth}}
{\tt\strut UPnPControlDeviceHandle} & for the open device, or 0 on failure
\end{tabular}}
\cxxSee{UPnP\_ControlOpenDevice,\ UPnP\_ControlOpenDeviceAsync
\strut}
\begin{cxxdoc}

Closes a device opened with UPnP\_ControlOpenDevice or
UPnP\_ControlOpenDeviceAsync ($\rightarrow$ 1.22, {\em page }\pageref{cxx.1.22}).  The device handle passed in may
not be used after this function is called.

Generates:
(no events)


\end{cxxdoc}
\end{cxxfunction}
\begin{cxxfunction}
{int}
        {UPnP\_ControlGetDeviceInfo}
        {(\ UPnPControlDeviceHandle\ device,\ \ UPnPControlDeviceInfo*\ info\ )}
        {Get information for an open device. }
        {1.25}
\cxxParameter{
\begin{tabular}[t]{lp{0.5\textwidth}}
{\tt\strut device} & handle returned byUPnP\_ControlOpenDevice 
\\
{\tt\strut info} & UPnPControlDeviceInfo to populate 
\end{tabular}}
\cxxReturn{
\begin{tabular}[t]{lp{0.5\textwidth}}
{\tt\strut error} & code
\end{tabular}}
\cxxSee{UPnP\_ControlGetServiceOwnerDeviceInfo
\strut}
\begin{cxxdoc}

Populates the fields of the specified UPnPControlDeviceInfo
structure with various information about the given device, such as
device type, UDN, manufacturer, model number, etc.

Generates:
(no events)


\end{cxxdoc}
\end{cxxfunction}
\begin{cxxfunction}
{int}
        {UPnP\_ControlGetServiceOwnerDeviceInfo}
        {(\ UPnPControlDeviceHandle\ handle,\ \ UPNP\_CHAR*\ serviceId,\ \ UPnPControlDeviceInfo*\ info\ )}
        {Get information for the parent device of a service }
        {1.26}
\cxxParameter{
\begin{tabular}[t]{lp{0.5\textwidth}}
{\tt\strut handle} & handle returned byUPnP\_ControlOpenDevice 
\\
{\tt\strut serviceId} & serviceId of service for whoseparent to get info 
\\
{\tt\strut info} & UPnPControlDeviceInfo topopulate 
\end{tabular}}
\cxxReturn{
\begin{tabular}[t]{lp{0.5\textwidth}}
{\tt\strut error} & code
\end{tabular}}
\cxxSee{UPnP\_ControlGetDeviceInfo,\ UPnP\_ControlGetServiceType
\strut}
\begin{cxxdoc}

Populates the fields of the specified UPnPControlDeviceInfo
structure with various information about the parent device of the
given service.  By contrast, UPnP\_ControlGetDeviceInfo gets
information about the root device ONLY.  This function is useful for
gathering information about embedded UPnP devices.

Generates:
(no events)


\end{cxxdoc}
\end{cxxfunction}
\begin{cxxfunction}
{UPNP\_CHAR*}
        {UPnP\_ControlGetServiceType}
        {(\ UPnPControlDeviceHandle\ deviceHandle,\ \ UPNP\_CHAR*\ serviceId\ )}
        {Get the type of a service }
        {1.27}
\cxxParameter{
\begin{tabular}[t]{lp{0.5\textwidth}}
{\tt\strut deviceHandle} & handle returned by UPnP\_ControlOpenDevice 
\\
{\tt\strut serviceId} & serviceId of service to gettype for 
\end{tabular}}
\cxxReturn{
\begin{tabular}[t]{lp{0.5\textwidth}}
{\tt\strut error} & code
\end{tabular}}
\cxxSee{UPnP\_ControlDeviceClose,\ UPnP\_ControlGetServiceOwnerDeviceInfo
\strut}
\begin{cxxdoc}

Returns the service type (as defined by the UPnP Forum) of the given
service instance on an open device.  The string passed back by this
function must not be modified in any way.  It is valid until
UPnP\_ControlDeviceClose is called on the given device handle.

Generates:
(no events)


\end{cxxdoc}
\end{cxxfunction}
\begin{cxxfunction}
{int}
        {UPnP\_ControlGetServices}
        {(\ UPnPControlDeviceHandle\ deviceHandle,\ \ UPnPControlServiceIterator*\ i\ )}
        {Enumerate the services on a device }
        {1.28}
\cxxParameter{
\begin{tabular}[t]{lp{0.5\textwidth}}
{\tt\strut deviceHandle} & handle returned by UPnP\_ControlOpenDevice 
\\
{\tt\strut i} & uninitialized UPnPControlServiceIteratorto use for this enumeration
\end{tabular}}
\cxxReturn{
\begin{tabular}[t]{lp{0.5\textwidth}}
{\tt\strut error} & code
\end{tabular}}
\cxxSee{UPnP\_ControlGetServicesByType,\ UPnP\_ControlNextService,
UPnP\_ControlServiceIteratorDone
\strut}
\begin{cxxdoc}

Initializes a UPnPControlServiceIterator to enumerate all the
services on all embedded devices on the given device.

Generates:
(no events)


\end{cxxdoc}
\end{cxxfunction}
\begin{cxxfunction}
{int}
        {UPnP\_ControlGetServicesByType}
        {(\ UPnPControlDeviceHandle\ deviceHandle,\ \ UPnPControlServiceIterator*\ i,\ \ UPNP\_CHAR*\ serviceType\ )}
        {Enumerate the services of a certain type on a device }
        {1.29}
\cxxParameter{
\begin{tabular}[t]{lp{0.5\textwidth}}
{\tt\strut deviceHandle} & handle returned by UPnP\_ControlOpenDevice 
\\
{\tt\strut i} & uninitialized UPnPControlServiceIteratorto use for this enumeration
\\
{\tt\strut serviceType} & type of service toenumerate 
\end{tabular}}
\cxxReturn{
\begin{tabular}[t]{lp{0.5\textwidth}}
{\tt\strut error} & code
\end{tabular}}
\cxxSee{UPnP\_ControlGetServices,\ UPnP\_ControlNextService,
UPnP\_ControlServiceIteratorDone
\strut}
\begin{cxxdoc}

Initializes a UPnPControlServiceIterator to enumerate all the
services of the given type (as defined by the UPnP Forum) on the given
device.

Generates:
(no events)


\end{cxxdoc}
\end{cxxfunction}
\begin{cxxfunction}
{UPNP\_CHAR*}
        {UPnP\_ControlNextService}
        {(\ UPnPControlServiceIterator*\ i\ )}
        {Enumerate the next service on the device. }
        {1.30}
\cxxParameter{
\begin{tabular}[t]{lp{0.5\textwidth}}
{\tt\strut i} & UPnPControlServiceIteratorinitialized by UPnP\_ControlGetServices or UPnP\_ControlGetServicesByType ($\rightarrow$ 1.28, {\em page }\pageref{cxx.1.28}) 
\end{tabular}}
\cxxReturn{
\begin{tabular}[t]{lp{0.5\textwidth}}
{\tt\strut serviceId} & if there is a next service; otherwise NULL
\end{tabular}}
\cxxSee{UPnP\_ControlGetServices,\ UPnP\_ControlGetServicesByType,
UPnP\_ControlServiceIteratorDone
\strut}
\begin{cxxdoc}

Returns the unique serviceId of the next service instance in the
given enumeration (initialized by UPnP\_ControlGetServices or
UPnP\_ControlGetServicesByType ($\rightarrow$ 1.28, {\em page }\pageref{cxx.1.28})).  The string returned by this
function must not be modified in any way, and is valid until
UPnP\_ControlDeviceClose is called for this device.

Generates:
(no events)


\end{cxxdoc}
\end{cxxfunction}
\begin{cxxfunction}
{void}
        {UPnP\_ControlServiceIteratorDone}
        {(\ UPnPControlServiceIterator*\ i\ )}
        {Clean up when done enumerating services. }
        {1.31}
\cxxParameter{
\begin{tabular}[t]{lp{0.5\textwidth}}
{\tt\strut i} & UPnPControlServiceIteratorinitialized by UPnP\_ControlGetServices or UPnP\_ControlGetServicesByType ($\rightarrow$ 1.28, {\em page }\pageref{cxx.1.28}) 
\end{tabular}}
\cxxReturn{
\begin{tabular}[t]{lp{0.5\textwidth}}
{\tt\strut nothing} & \end{tabular}}
\cxxSee{UPnP\_ControlGetServices,\ UPnP\_ControlGetServicesByType,
UPnP\_ControlNextService
\strut}
\begin{cxxdoc}

Must be called when done enumerating services using
UPnP\_ControlNextService.

Generates:
(no events)


\end{cxxdoc}
\end{cxxfunction}
\begin{cxxfunction}
{IXML\_Document*}
        {UPnP\_ControlGetServiceInfo}
        {(\ UPnPControlDeviceHandle\ deviceHandle,\ \ UPNP\_CHAR*\ serviceId\ )}
        {Get detailed information about a service. }
        {1.32}
\cxxParameter{
\begin{tabular}[t]{lp{0.5\textwidth}}
{\tt\strut deviceHandle} & handle returned by UPnP\_ControlOpenDevice 
\\
{\tt\strut serviceId} & id of service to get detailedinfo for 
\end{tabular}}
\cxxReturn{
\begin{tabular}[t]{lp{0.5\textwidth}}
{\tt\strut XML} & DOM tree for the given document, or NULL on error
\end{tabular}}
\cxxSee{UPnP\_ControlGetServiceInfoAsync
\strut}
\begin{cxxdoc}

This function retrieves the service description document(SCPD) from the
given device and parses it into an XML DOM structure.  The resulting
IXML\_Document returned by this function is owned by the caller of this
function.  See the UPnP Forum SCPD definition for more information on
the content and structure of this document.

When this function returns, the document has either successfully loaded,
or an error has occurred

Generates:
(no events)


\end{cxxdoc}
\end{cxxfunction}
\begin{cxxfunction}
{int}
        {UPnP\_ControlGetServiceInfoAsync}
        {(\ UPnPControlDeviceHandle\ deviceHandle,\ \ UPNP\_CHAR*\ serviceId,\ \ void*\ userData\ )}
        {Asynchronous get detailed information about a service. }
        {1.33}
\cxxParameter{
\begin{tabular}[t]{lp{0.5\textwidth}}
{\tt\strut deviceHandle} & handle returned by UPnP\_ControlOpenDevice 
\\
{\tt\strut serviceId} & id of service to get detailedinfo for 
\\
{\tt\strut userData} & passed to callback asuserRequestData for generatedevents 
\end{tabular}}
\cxxReturn{
\begin{tabular}[t]{lp{0.5\textwidth}}
{\tt\strut error} & code
\end{tabular}}
\cxxSee{UPnP\_ControlGetServiceInfo
\strut}
\begin{cxxdoc}

This function retrieves the service description document (SCPD) from
the given device and parses it into an XML DOM structure.  The
IXML\_Document is passed back to the control point through a
UPNP\_CONTROL\_EVENT\_SERVICE\_INFO\_READ event.

This function will return immediately; an event from the list below
is sent to the control point when the SCPD has been downloaded, or the
operation fails due to some error.

Generates:
UPNP\_CONTROL\_EVENT\_SERVICE\_INFO\_READ,
UPNP\_CONTROL\_EVENT\_SERVICE\_GET\_INFO\_FAILED


\end{cxxdoc}
\end{cxxfunction}
\begin{cxxfunction}
{int}
        {UPnP\_ControlInvokeAction}
        {(\ UPnPControlDeviceHandle\ deviceHandle,\ \ UPNP\_CHAR*\ serviceId,\ \ const\ UPNP\_CHAR*\ actionName,\ \ IXML\_Document*\ action,\ \ void*\ userData,\ \ UPNP\_BOOL\ waitForCompletion\ )}
        {Invoke an action on a remote service }
        {1.34}
\cxxParameter{
\begin{tabular}[t]{lp{0.5\textwidth}}
{\tt\strut deviceHandle} & handle returned by UPnP\_ControlOpenDevice 
\\
{\tt\strut serviceId} & id of service to invokeaction on 
\\
{\tt\strut actionName} & name of action to invoke 
\\
{\tt\strut action} & SOAP action message createdusing UPnP\_CreateActionand UPnP\_SetActionArg 
\\
{\tt\strut userData} & passed to callback asuserRequestData for generatedevents 
\\
{\tt\strut waitForCompletion} & see above description 
\end{tabular}}
\cxxReturn{
\begin{tabular}[t]{lp{0.5\textwidth}}
{\tt\strut error} & code
\end{tabular}}
\cxxSee{\strut}
\begin{cxxdoc}

This function will send the given action to a service on a remote
device and generate an event that contains the response to that action.
If waitForCompletion is UPNP\_TRUE, then UPnP\_ControlInvokeAction will
not return until the UPNP\_CONTROL\_EVENT\_ACTION\_COMPLETE event is
passed to the control point callback.  Otherwise, it will return
immediately and the event is sent to the control point when the
action completes or an error occurs.

The action passed into this function must be generated using
UPnP\_CreateAction and UPnP\_SetActionArg.

Generates:
UPNP\_CONTROL\_EVENT\_ACTION\_COMPLETE


\end{cxxdoc}
\end{cxxfunction}
\begin{cxxfunction}
{int}
        {UPnP\_ControlSubscribe}
        {(\ UPnPControlDeviceHandle\ deviceHandle,\ \ UPNP\_CHAR*\ serviceId,\ \ UPNP\_INT32\ timeoutSec,\ \ void*\ userData,\ \ UPNP\_BOOL\ waitForCompletion\ )}
        {Subscribe to a service or renew a subscription }
        {1.35}
\cxxParameter{
\begin{tabular}[t]{lp{0.5\textwidth}}
{\tt\strut deviceHandle} & handle returned by UPnP\_ControlOpenDevice 
\\
{\tt\strut serviceId} & id of service to subscribeto 
\\
{\tt\strut timeoutSec} & duration in seconds for thesubscription 
\\
{\tt\strut userData} & passed to callback asuserRequestData for generatedevents 
\\
{\tt\strut waitForCompletion} & see above description 
\end{tabular}}
\cxxReturn{
\begin{tabular}[t]{lp{0.5\textwidth}}
{\tt\strut error} & code
\end{tabular}}
\cxxSee{UPnP\_ControlUnsubscribe,\ UPnP\_ControlSubscribedToService
\strut}
\begin{cxxdoc}

Subscribes to a service on a remote device to receive notifications
when the service's state changes.  If the control point is already
subscribed to the given service, this function has the effect of
renewing the existing subscription for the given period of time.

If waitForCompletion is UPNP\_TRUE, this function will wait until the
subscription request has been either accepted or rejected.  Otherwise
it returns immediately and a UPNP\_CONTROL\_EVENT\_SUBSCRIPTION\_ACCEPTED
or UPNP\_CONTROL\_EVENT\_SUBSCRIPTION\_REJECTED event is generated once
the device responds.

Once the control point is subscribed to service,
UPNP\_CONTROL\_EVENT\_SERVICE\_STATE\_UPDATE	events may be generated to
indicate the service's state has been updated. If such a notification
is dropped, a UPNP\_CONTROL\_EVENT\_SUBSCRIPTION\_OUT\_OF\_SYNC event will be
sent to the control point, which gives the application the opportunity
to indicate through the callback return value whether to drop the
subscription or attempt to re-subscribe.  If the application chooses to
re-subscribe (synchronize), a UPNP\_CONTROL\_EVENT\_SYNCHRONIZE\_FAILED
event may be generated, if the synchronization fails.  If the
UPNP\_CONTROL\_EVENT\_SYNCHRONIZE\_FAILED is not handled by the control
point callback, the default action is to try to synchronize the
subscription by re-subscribing.

When the subscription is close to expiring, the control point will
receive a UPNP\_CONTROL\_EVENT\_SUBSCRIPTION\_NEAR\_EXPIRATION event, to
give the application a chance to renew the subscription.

Once subscription expires, a UPNP\_CONTROL\_EVENT\_SUBSCRIPTION\_EXPIRED
event is sent to the control point.

Generates:
UPNP\_CONTROL\_EVENT\_SUBSCRIPTION\_ACCEPTED,
UPNP\_CONTROL\_EVENT\_SUBSCRIPTION\_REJECTED,
UPNP\_CONTROL\_EVENT\_SERVICE\_STATE\_UPDATE,
UPNP\_CONTROL\_EVENT\_SUBSCRIPTION\_OUT\_OF\_SYNC,
UPNP\_CONTROL\_EVENT\_SYNCHRONIZE\_FAILED,
UPNP\_CONTROL\_EVENT\_SUBSCRIPTION\_NEAR\_EXPIRATION,
UPNP\_CONTROL\_EVENT\_SUBSCRIPTION\_EXPIRED


\end{cxxdoc}
\end{cxxfunction}
\begin{cxxfunction}
{UPNP\_BOOL}
        {UPnP\_ControlSubscribedToService}
        {(\ UPnPControlDeviceHandle\ deviceHandle,\ \ UPNP\_CHAR*\ serviceId\ )}
        {Return whether or not the control point is subscribed to the given service }
        {1.36}
\cxxParameter{
\begin{tabular}[t]{lp{0.5\textwidth}}
{\tt\strut deviceHandle} & handle returned by UPnP\_ControlOpenDevice 
\\
{\tt\strut serviceId} & id of service 
\end{tabular}}
\cxxReturn{
\begin{tabular}[t]{lp{0.5\textwidth}}
{\tt\strut UPNP\_TRUE} & if subscribed, UPNP\_FALSE otherwise
\end{tabular}}
\cxxSee{UPnP\_ControlSubscribe,\ UPnP\_ControlUnsubscribe
\strut}
\begin{cxxdoc}


\end{cxxdoc}
\end{cxxfunction}
\begin{cxxfunction}
{void}
        {UPnP\_ControlSetServiceSubscriptionExpireWarning}
        {(\ UPnPControlDeviceHandle\ deviceHandle,\ \ UPNP\_CHAR*\ serviceId,\ \ UPNP\_INT32\ warningMsec\ )}
        {Sets the time offset before subscription expiration at which a warning event will be generated. }
        {1.37}
\cxxParameter{
\begin{tabular}[t]{lp{0.5\textwidth}}
{\tt\strut deviceHandle} & handle returned by UPnP\_ControlOpenDevice 
\\
{\tt\strut serviceId} & id of service 
\\
{\tt\strut warningMsec} & time offset to generatewarning, in milliseconds 
\end{tabular}}
\cxxReturn{
\begin{tabular}[t]{lp{0.5\textwidth}}
{\tt\strut nothing} & \end{tabular}}
\cxxSee{UPnP\_ControlSubscribe,\ UPnP\_ControlUnsubscribe
\strut}
\begin{cxxdoc}

This function should be called, if desired, before subscribing to
the given service.


\end{cxxdoc}
\end{cxxfunction}
\begin{cxxfunction}
{int}
        {UPnP\_ControlUnsubscribe}
        {(\ UPnPControlDeviceHandle\ deviceHandle,\ \ UPNP\_CHAR*\ serviceId,\ \ void*\ userData,\ \ UPNP\_BOOL\ waitForCompletion\ )}
        {Cancel a subscribed service }
        {1.38}
\cxxParameter{
\begin{tabular}[t]{lp{0.5\textwidth}}
{\tt\strut deviceHandle} & handle returned by UPnP\_ControlOpenDevice 
\\
{\tt\strut serviceId} & id of service to unsubscribefrom 
\\
{\tt\strut userData} & passed to callback asuserRequestData for generatedevents 
\\
{\tt\strut waitForCompletion} & see above description 
\end{tabular}}
\cxxReturn{
\begin{tabular}[t]{lp{0.5\textwidth}}
{\tt\strut error} & code
\end{tabular}}
\cxxSee{\strut}
\begin{cxxdoc}

If waitForCompletion is UPNP\_TRUE, this function waits for the
unsubscribe operation to complete before returning.  Otherwise, it
returns immediately and a UPNP\_CONTROL\_EVENT\_SUBSCRIPTION\_CANCELLED
event is generated when the operation completes.

Generates:
UPNP\_CONTROL\_EVENT\_SUBSCRIPTION\_CANCELLED


\end{cxxdoc}
\end{cxxfunction}
\end{cxxentry}
\begin{cxxentry}
{}
        {Internal Library Documentation}
        {}
        {}
        {2}
\begin{cxxnames}
\cxxitem{}
        {Discovery/SSDP}
        {}
        {}
        {2.1}
\cxxitem{}
        {Description}
        {}
        {}
        {2.2}
\cxxitem{}
        {Control/SOAP}
        {}
        {}
        {2.3}
\cxxitem{}
        {Eventing/GENA}
        {}
        {}
        {2.4}
\end{cxxnames}
\begin{cxxdoc}
These functions are \<b\>not\</b\> at the API level and should not be called from
code outside the UPnP library. 
\end{cxxdoc}
\begin{cxxentry}
{}
        {Discovery/SSDP}
        {}
        {}
        {2.1}
\begin{cxxnames}
\cxxitem{SSDP\_INT32}
        {SSDP\_ServerInit}
        {(\ SSDPServerContext*\ ctx,\ \ SSDP\_UINT8*\ ipAddr,\ \ SSDP\_INT16\ ipType,\ \ const\ SSDP\_CHAR*\ serverId,\ \ SSDPCallback\ cb,\ \ void*\ cookie)}
        {SSDP server initialization routine. }
        {2.1.1}
\cxxitem{SSDP\_INT32}
        {SSDP\_ServerAddToSelectList}
        {(\ SSDPServerContext*\ ctx,\ RTP\_FD\_SET*\ readList,\ RTP\_FD\_SET*\ writeList,\ RTP\_FD\_SET*\ errList\ )}
        {}
        {2.1.2}
\cxxitem{SSDP\_BOOL}
        {SSDP\_ServerProcessState}
        {(\ SSDPServerContext*\ ctx,\ RTP\_FD\_SET*\ readList,\ RTP\_FD\_SET*\ writeList,\ RTP\_FD\_SET*\ errList\ )}
        {}
        {2.1.3}
\cxxitem{void}
        {SSDP\_ServerDestroy}
        {(\ SSDPServerContext*\ ctx\ pointer\ to\ SSDP\ context*\ /\ )}
        {}
        {2.1.4}
\cxxitem{SSDP\_INT32}
        {SSDP\_SendNotify}
        {(\ SSDPServerContext*\ ctx,\ \ const\ SSDP\_CHAR*\ notifyType,\ \ const\ SSDP\_CHAR*\ notifySubType,\ \ const\ SSDP\_CHAR*\ usn,\ \ const\ SSDP\_CHAR*\ location,\ \ SSDP\_UINT32*\ timeout)}
        {Send a SSDP notification for the device or service. }
        {2.1.5}
\cxxitem{SSDP\_INT32}
        {SSDP\_SendResponse}
        {(\ SSDPServerContext*\ ctx,\ \ SSDPPendingResponse*\ response)}
        {Deliver a responce to SSDP discovery request.}
        {2.1.6}
\cxxitem{SSDP\_INT32}
        {\_SSDP\_ProcessOneRequest}
        {(\ SSDPServerContext*\ ctx\ )}
        {Process an incoming SSDP discovery request. }
        {2.1.7}
\cxxitem{int}
        {SSDP\_ParseRequest}
        {(\ SSDPServerContext*\ ctx,\ \ SSDPServerRequest*\ ssdpRequest\ )}
        {Extract SSDP request. }
        {2.1.8}
\cxxitem{int}
        {SSDP\_McastRead}
        {(\ void*\ cookie,\ \ SSDP\_UINT8*\ buffer,\ \ SSDP\_INT32\ min,\ \ SSDP\_INT32\ max)}
        {Reads all messages posted to the multicast address}
        {2.1.9}
\cxxitem{int}
        {\_SSDP\_ReadMSearchHeader}
        {(\ void*\ request,\ \ HTTPSession*\ ptr,\ \ HTTPHeaderType\ type,\ \ const\ HTTP\_CHAR*\ name,\ \ const\ HTTP\_CHAR*\ value\ )}
        {Extracts MX and St headers from a SSDP request}
        {2.1.10}
\cxxitem{int}
        {\_SSDP\_ReadNotifyHeader}
        {(\ void*\ request,\ \ HTTPSession*\ ptr,\ \ HTTPHeaderType\ type,\ \ const\ HTTP\_CHAR*\ name,\ \ const\ HTTP\_CHAR*\ value\ )}
        {Extracts MX and St headers from a SSDP request}
        {2.1.11}
\cxxitem{SSDP\_INT32}
        {SSDP\_QueueSearchResponse}
        {(\ SSDPServerContext*\ ctx,\ \ SSDPSearch*\ search,\ \ const\ SSDP\_CHAR*\ targetLocation,\ \ const\ SSDP\_CHAR*\ targetURN,\ \ SSDP\_UINT32\ targetTimeoutSec)}
        {Queues a response to the response list based on its scheduled delivery time.}
        {2.1.12}
\cxxitem{SSDP\_INT32}
        {SSDP\_CheckPendingResponses}
        {(\ SSDPServerContext*\ ctx,\ \ SSDP\_UINT32\ currentTimeMsec)}
        {Delivers responses scheduled for delivery. }
        {2.1.13}
\cxxitem{void}
        {SSDP\_ProcessError}
        {(\ SSDP\_CHAR*\ errMsg)}
        {Process SSDP Errors}
        {2.1.14}
\cxxitem{SSDP\_UINT32}
        {SSDP\_RandMax}
        {(\ SSDP\_UINT32\ mxLimit)}
        {generates a random number between 0 and mxLimit}
        {2.1.15}
\cxxitem{int}
        {SSDP\_SearchInit}
        {(\ SSDPClientSearch*\ search,\ HTTPManagedClient*\ httpClient,\ SSDP\_INT16\ ipType,\ SSDP\_CHAR*\ searchType,\ SSDP\_INT32\ maxResponseTimeoutSec,\ SSDPSearchCallback\ callbackFn,\ void*\ callbackData\ )}
        {}
        {2.1.16}
\cxxitem{void}
        {SSDP\_SearchDestroy}
        {(\ SSDPClientSearch*\ search\ )}
        {}
        {2.1.17}
\cxxitem{int}
        {SSDP\_SearchExecute}
        {(\ SSDPClientSearch*\ search\ )}
        {}
        {2.1.18}
\cxxitem{SSDP\_INT32}
        {SSDP\_SearchAddToSelectList}
        {(\ SSDPClientSearch*\ search,\ RTP\_FD\_SET*\ readList,\ RTP\_FD\_SET*\ writeList,\ RTP\_FD\_SET*\ errList\ )}
        {}
        {2.1.19}
\cxxitem{SSDP\_BOOL}
        {SSDP\_SearchProcessState}
        {(\ SSDPClientSearch*\ search,\ RTP\_FD\_SET*\ readList,\ RTP\_FD\_SET*\ writeList,\ RTP\_FD\_SET*\ errList\ )}
        {}
        {2.1.20}
\end{cxxnames}
\begin{cxxdoc}
These functions are \<b\>not\</b\> at the API level and should not be called from
code outside the UPnP library. 
\end{cxxdoc}
\begin{cxxfunction}
{SSDP\_INT32}
        {SSDP\_ServerInit}
        {(\ SSDPServerContext*\ ctx,\ \ SSDP\_UINT8*\ ipAddr,\ \ SSDP\_INT16\ ipType,\ \ const\ SSDP\_CHAR*\ serverId,\ \ SSDPCallback\ cb,\ \ void*\ cookie)}
        {SSDP server initialization routine. }
        {2.1.1}
\cxxParameter{
\begin{tabular}[t]{lp{0.5\textwidth}}
{\tt\strut ctx} & pointer to an uninitialized SSDP context structure 
\\
{\tt\strut ipAddr} & IP address of the host to bind the UDP socket to, if a NULL issupplied, the UDP socket is bound to the local IP address 
\\
{\tt\strut ipType} & ip version 4 or ipversion 6 
\\
{\tt\strut serverId} & String holding platform name 
\\
{\tt\strut cb} & SSDP Callback routine 
\\
{\tt\strut cookie} & cookie(runtime) to be stored in ssdp context to be passedlater to ssdp callback 
\end{tabular}}
\cxxReturn{
\begin{tabular}[t]{lp{0.5\textwidth}}
{\tt\strut error} & code
\end{tabular}}
\begin{cxxdoc}
This routine starts up SSDP services by creating a UDP socket, getting the ssdp
multicast membership for the socket and initlializing ssdp context.


\end{cxxdoc}
\end{cxxfunction}
\begin{cxxfunction}
{SSDP\_INT32}
        {SSDP\_ServerAddToSelectList}
        {(\ SSDPServerContext*\ ctx,\ RTP\_FD\_SET*\ readList,\ RTP\_FD\_SET*\ writeList,\ RTP\_FD\_SET*\ errList\ )}
        {}
        {2.1.2}
\end{cxxfunction}
\begin{cxxfunction}
{SSDP\_BOOL}
        {SSDP\_ServerProcessState}
        {(\ SSDPServerContext*\ ctx,\ RTP\_FD\_SET*\ readList,\ RTP\_FD\_SET*\ writeList,\ RTP\_FD\_SET*\ errList\ )}
        {}
        {2.1.3}
\end{cxxfunction}
\begin{cxxfunction}
{void}
        {SSDP\_ServerDestroy}
        {(\ SSDPServerContext*\ ctx\ pointer\ to\ SSDP\ context*\ /\ )}
        {}
        {2.1.4}
\end{cxxfunction}
\begin{cxxfunction}
{SSDP\_INT32}
        {SSDP\_SendNotify}
        {(\ SSDPServerContext*\ ctx,\ \ const\ SSDP\_CHAR*\ notifyType,\ \ const\ SSDP\_CHAR*\ notifySubType,\ \ const\ SSDP\_CHAR*\ usn,\ \ const\ SSDP\_CHAR*\ location,\ \ SSDP\_UINT32*\ timeout)}
        {Send a SSDP notification for the device or service. }
        {2.1.5}
\cxxParameter{
\begin{tabular}[t]{lp{0.5\textwidth}}
{\tt\strut ctx} & pointer to SSDP context 
\\
{\tt\strut notifyType} & the notification type (NT) string 
\\
{\tt\strut notifySubType} & pointer to string containing NT subtype 
\\
{\tt\strut usn} & pointer to string containing USN header 
\\
{\tt\strut location} & pointer to string containing Location header 
\\
{\tt\strut timeout} & pointer to max-age header value 
\end{tabular}}
\cxxReturn{
\begin{tabular}[t]{lp{0.5\textwidth}}
{\tt\strut error} & code
\end{tabular}}
\begin{cxxdoc}
This routine sends ssdp alive and bye-bye notifications on the mulicast
address for devices and servces


\end{cxxdoc}
\end{cxxfunction}
\begin{cxxfunction}
{SSDP\_INT32}
        {SSDP\_SendResponse}
        {(\ SSDPServerContext*\ ctx,\ \ SSDPPendingResponse*\ response)}
        {Deliver a responce to SSDP discovery request.}
        {2.1.6}
\cxxParameter{
\begin{tabular}[t]{lp{0.5\textwidth}}
{\tt\strut ctx} & pointer to SSDP context 
\\
{\tt\strut response} & buffer holding response information 
\end{tabular}}
\cxxReturn{
\begin{tabular}[t]{lp{0.5\textwidth}}
{\tt\strut error} & code
\end{tabular}}
\begin{cxxdoc}
Deliver a responce to SSDP discovery request.


\end{cxxdoc}
\end{cxxfunction}
\begin{cxxfunction}
{SSDP\_INT32}
        {\_SSDP\_ProcessOneRequest}
        {(\ SSDPServerContext*\ ctx\ )}
        {Process an incoming SSDP discovery request. }
        {2.1.7}
\cxxParameter{
\begin{tabular}[t]{lp{0.5\textwidth}}
{\tt\strut ctx} & pointer to SSDP context 
\end{tabular}}
\cxxReturn{
\begin{tabular}[t]{lp{0.5\textwidth}}
{\tt\strut error} & code
\end{tabular}}
\begin{cxxdoc}
Processes a SSDP discovery request available during a period given by
timeoutMsec milli seconds.


\end{cxxdoc}
\end{cxxfunction}
\begin{cxxfunction}
{int}
        {SSDP\_ParseRequest}
        {(\ SSDPServerContext*\ ctx,\ \ SSDPServerRequest*\ ssdpRequest\ )}
        {Extract SSDP request. }
        {2.1.8}
\cxxParameter{
\begin{tabular}[t]{lp{0.5\textwidth}}
{\tt\strut ctx} & pointer to SSDP context 
\\
{\tt\strut ssdpRequest} & address of the SSDPServerEvent structure to fill up 
\end{tabular}}
\cxxReturn{
\begin{tabular}[t]{lp{0.5\textwidth}}
{\tt\strut error} & code
\end{tabular}}
\begin{cxxdoc}
Retrieves messages from the multicast address, if ssdp request is
detected a buffer holding request information is populated


\end{cxxdoc}
\end{cxxfunction}
\begin{cxxfunction}
{int}
        {SSDP\_McastRead}
        {(\ void*\ cookie,\ \ SSDP\_UINT8*\ buffer,\ \ SSDP\_INT32\ min,\ \ SSDP\_INT32\ max)}
        {Reads all messages posted to the multicast address}
        {2.1.9}
\cxxParameter{
\begin{tabular}[t]{lp{0.5\textwidth}}
{\tt\strut cookie} & internal cookie 
\\
{\tt\strut buffer} & pointer to buffer contaning request message 
\\
{\tt\strut min} & \\
{\tt\strut max} & max size to be read 
\end{tabular}}
\cxxReturn{
\begin{tabular}[t]{lp{0.5\textwidth}}
{\tt\strut error} & code
\end{tabular}}
\begin{cxxdoc}
Reads all messages posted to the multicast address


\end{cxxdoc}
\end{cxxfunction}
\begin{cxxfunction}
{int}
        {\_SSDP\_ReadMSearchHeader}
        {(\ void*\ request,\ \ HTTPSession*\ ptr,\ \ HTTPHeaderType\ type,\ \ const\ HTTP\_CHAR*\ name,\ \ const\ HTTP\_CHAR*\ value\ )}
        {Extracts MX and St headers from a SSDP request}
        {2.1.10}
\cxxParameter{
\begin{tabular}[t]{lp{0.5\textwidth}}
{\tt\strut request} & request buffer to be populated 
\\
{\tt\strut ptr} & current HTTP session 
\\
{\tt\strut type} & HTTP header type 
\\
{\tt\strut name} & holds the name of the header 
\\
{\tt\strut value} & holds the value of the header
\end{tabular}}
\cxxReturn{
\begin{tabular}[t]{lp{0.5\textwidth}}
{\tt\strut error} & code
\end{tabular}}
\begin{cxxdoc}
Extracts MX and St headers from a SSDP request


\end{cxxdoc}
\end{cxxfunction}
\begin{cxxfunction}
{int}
        {\_SSDP\_ReadNotifyHeader}
        {(\ void*\ request,\ \ HTTPSession*\ ptr,\ \ HTTPHeaderType\ type,\ \ const\ HTTP\_CHAR*\ name,\ \ const\ HTTP\_CHAR*\ value\ )}
        {Extracts MX and St headers from a SSDP request}
        {2.1.11}
\cxxParameter{
\begin{tabular}[t]{lp{0.5\textwidth}}
{\tt\strut request} & request buffer to be populated 
\\
{\tt\strut ptr} & current HTTP session 
\\
{\tt\strut type} & HTTP header type 
\\
{\tt\strut name} & holds the name of the header 
\\
{\tt\strut value} & holds the value of the header
\end{tabular}}
\cxxReturn{
\begin{tabular}[t]{lp{0.5\textwidth}}
{\tt\strut error} & code
\end{tabular}}
\begin{cxxdoc}
Extracts MX and St headers from a SSDP request


\end{cxxdoc}
\end{cxxfunction}
\begin{cxxfunction}
{SSDP\_INT32}
        {SSDP\_QueueSearchResponse}
        {(\ SSDPServerContext*\ ctx,\ \ SSDPSearch*\ search,\ \ const\ SSDP\_CHAR*\ targetLocation,\ \ const\ SSDP\_CHAR*\ targetURN,\ \ SSDP\_UINT32\ targetTimeoutSec)}
        {Queues a response to the response list based on its scheduled delivery time.}
        {2.1.12}
\cxxParameter{
\begin{tabular}[t]{lp{0.5\textwidth}}
{\tt\strut ctx} & pointer to SSDP context 
\\
{\tt\strut search} & pointer to the buffer containingthe request information 
\\
{\tt\strut targetLocation} & pointer to string containing Location header
\\
{\tt\strut targetURN} & pointer to string containing USN header 
\\
{\tt\strut targetTimeoutSec} & max-age header value 
\end{tabular}}
\cxxReturn{
\begin{tabular}[t]{lp{0.5\textwidth}}
{\tt\strut error} & code
\end{tabular}}
\begin{cxxdoc}
Queues a response to the response list based on its scheduled delivery time.
A random delivery time within targetTimeoutSec duration is calculated.
This response is positioned in the list according to its scheduled delivery time.


\end{cxxdoc}
\end{cxxfunction}
\begin{cxxfunction}
{SSDP\_INT32}
        {SSDP\_CheckPendingResponses}
        {(\ SSDPServerContext*\ ctx,\ \ SSDP\_UINT32\ currentTimeMsec)}
        {Delivers responses scheduled for delivery. }
        {2.1.13}
\cxxParameter{
\begin{tabular}[t]{lp{0.5\textwidth}}
{\tt\strut ctx} & pointer to SSDP context 
\\
{\tt\strut currentTimeMsec} & time against which the scheduledtime is checked 
\end{tabular}}
\cxxReturn{
\begin{tabular}[t]{lp{0.5\textwidth}}
{\tt\strut error} & code
\end{tabular}}
\begin{cxxdoc}
Scan the pending response list and deliver responses for which the
scheduled time count is less than supplied time count.


\end{cxxdoc}
\end{cxxfunction}
\begin{cxxfunction}
{void}
        {SSDP\_ProcessError}
        {(\ SSDP\_CHAR*\ errMsg)}
        {Process SSDP Errors}
        {2.1.14}
\cxxParameter{
\begin{tabular}[t]{lp{0.5\textwidth}}
{\tt\strut errMsg} & error message string 
\end{tabular}}
\cxxReturn{
\begin{tabular}[t]{lp{0.5\textwidth}}
{\tt\strut None} & \end{tabular}}
\begin{cxxdoc}
Process SSDP Errors


\end{cxxdoc}
\end{cxxfunction}
\begin{cxxfunction}
{SSDP\_UINT32}
        {SSDP\_RandMax}
        {(\ SSDP\_UINT32\ mxLimit)}
        {generates a random number between 0 and mxLimit}
        {2.1.15}
\cxxParameter{
\begin{tabular}[t]{lp{0.5\textwidth}}
{\tt\strut mxLimit} & upper limit of the random number 
\end{tabular}}
\cxxReturn{
\begin{tabular}[t]{lp{0.5\textwidth}}
{\tt\strut error} & code
\end{tabular}}
\begin{cxxdoc}
generates a random number between 0 and mxLimit


\end{cxxdoc}
\end{cxxfunction}
\begin{cxxfunction}
{int}
        {SSDP\_SearchInit}
        {(\ SSDPClientSearch*\ search,\ HTTPManagedClient*\ httpClient,\ SSDP\_INT16\ ipType,\ SSDP\_CHAR*\ searchType,\ SSDP\_INT32\ maxResponseTimeoutSec,\ SSDPSearchCallback\ callbackFn,\ void*\ callbackData\ )}
        {}
        {2.1.16}
\cxxReturn{
\begin{tabular}[t]{lp{0.5\textwidth}}
{\tt\strut error} & code
\end{tabular}}
\end{cxxfunction}
\begin{cxxfunction}
{void}
        {SSDP\_SearchDestroy}
        {(\ SSDPClientSearch*\ search\ )}
        {}
        {2.1.17}
\cxxReturn{
\begin{tabular}[t]{lp{0.5\textwidth}}
{\tt\strut error} & code
\end{tabular}}
\end{cxxfunction}
\begin{cxxfunction}
{int}
        {SSDP\_SearchExecute}
        {(\ SSDPClientSearch*\ search\ )}
        {}
        {2.1.18}
\cxxReturn{
\begin{tabular}[t]{lp{0.5\textwidth}}
{\tt\strut error} & code
\end{tabular}}
\end{cxxfunction}
\begin{cxxfunction}
{SSDP\_INT32}
        {SSDP\_SearchAddToSelectList}
        {(\ SSDPClientSearch*\ search,\ RTP\_FD\_SET*\ readList,\ RTP\_FD\_SET*\ writeList,\ RTP\_FD\_SET*\ errList\ )}
        {}
        {2.1.19}
\cxxReturn{
\begin{tabular}[t]{lp{0.5\textwidth}}
{\tt\strut error} & code
\end{tabular}}
\end{cxxfunction}
\begin{cxxfunction}
{SSDP\_BOOL}
        {SSDP\_SearchProcessState}
        {(\ SSDPClientSearch*\ search,\ RTP\_FD\_SET*\ readList,\ RTP\_FD\_SET*\ writeList,\ RTP\_FD\_SET*\ errList\ )}
        {}
        {2.1.20}
\cxxReturn{
\begin{tabular}[t]{lp{0.5\textwidth}}
{\tt\strut error} & code
\end{tabular}}
\end{cxxfunction}
\end{cxxentry}
\begin{cxxentry}
{}
        {Description}
        {}
        {}
        {2.2}
\begin{cxxdoc}
These functions are \<b\>not\</b\> at the API level and should not be called from
code outside the UPnP library. 
\end{cxxdoc}
\end{cxxentry}
\begin{cxxentry}
{}
        {Control/SOAP}
        {}
        {}
        {2.3}
\begin{cxxnames}
\cxxitem{SOAP\_INT32}
        {SOAP\_ActionInit}
        {(\ SOAPAction*\ action,\ HTTPManagedClient*\ httpClient,\ SOAP\_INT16\ ipType,\ const\ SOAP\_CHAR*\ destUri,\ const\ SOAP\_CHAR*\ soapAction,\ SOAP\_CHAR*\ headerStr,\ SOAP\_INT32\ headerLen,\ SOAP\_CHAR*\ bodyStr,\ SOAP\_INT32\ bodyLen,\ SOAPActionCallback\ callbackFn,\ void*\ callbackData\ )}
        {}
        {2.3.1}
\cxxitem{SOAP\_INT32}
        {SOAP\_ActionInitEx}
        {(\ SOAPAction*\ action,\ HTTPManagedClient*\ httpClient,\ SOAP\_INT16\ ipType,\ URLDescriptor*\ postURL,\ URLDescriptor*\ baseURL,\ const\ SOAP\_CHAR*\ soapAction,\ SOAP\_CHAR*\ headerStr,\ SOAP\_INT32\ headerLen,\ SOAP\_CHAR*\ bodyStr,\ SOAP\_INT32\ bodyLen,\ SOAPActionCallback\ callbackFn,\ void*\ callbackData\ )}
        {}
        {2.3.2}
\cxxitem{SOAP\_INT32}
        {SOAP\_CreateHttpSession}
        {(\ SOAPAction*\ action\ )}
        {}
        {2.3.3}
\cxxitem{void}
        {SOAP\_ActionDestroy}
        {(\ SOAPAction*\ action\ )}
        {}
        {2.3.4}
\cxxitem{SOAP\_INT32}
        {SOAP\_ActionExecute}
        {(\ SOAPAction*\ action\ )}
        {}
        {2.3.5}
\cxxitem{SOAP\_INT32}
        {SOAP\_ActionAddToSelectList}
        {(\ SOAPAction*\ action,\ RTP\_FD\_SET*\ readList,\ RTP\_FD\_SET*\ writeList,\ RTP\_FD\_SET*\ errList\ )}
        {}
        {2.3.6}
\cxxitem{SOAP\_BOOL}
        {SOAP\_ActionProcessState}
        {(\ SOAPAction*\ action,\ RTP\_FD\_SET*\ readList,\ RTP\_FD\_SET*\ writeList,\ RTP\_FD\_SET*\ errList\ )}
        {}
        {2.3.7}
\cxxitem{SOAP\_INT32}
        {SOAP\_SendActionRequest}
        {(\ SOAPAction*\ action\ )}
        {}
        {2.3.8}
\cxxitem{SOAP\_INT32}
        {SOAP\_ReadActionResponse}
        {(\ SOAPAction*\ action\ )}
        {}
        {2.3.9}
\cxxitem{SOAPActionState}
        {\_SOAP\_ParseActionRespose}
        {(\ HTTPResponseInfo*\ info,\ HTTPManagedClientSession*\ session,\ SOAPActionResponse*\ response\ )}
        {}
        {2.3.10}
\end{cxxnames}
\begin{cxxdoc}
These functions are \<b\>not\</b\> at the API level and should not be called from
code outside the UPnP library. 
\end{cxxdoc}
\begin{cxxfunction}
{SOAP\_INT32}
        {SOAP\_ActionInit}
        {(\ SOAPAction*\ action,\ HTTPManagedClient*\ httpClient,\ SOAP\_INT16\ ipType,\ const\ SOAP\_CHAR*\ destUri,\ const\ SOAP\_CHAR*\ soapAction,\ SOAP\_CHAR*\ headerStr,\ SOAP\_INT32\ headerLen,\ SOAP\_CHAR*\ bodyStr,\ SOAP\_INT32\ bodyLen,\ SOAPActionCallback\ callbackFn,\ void*\ callbackData\ )}
        {}
        {2.3.1}
\cxxReturn{
\begin{tabular}[t]{lp{0.5\textwidth}}
{\tt\strut error} & code
\end{tabular}}
\begin{cxxdoc}
desc::: this will be called each time a soap action is to be performed. This will
open a http client, and send an POST, the request structure is filled and
send to be upnp client manager to add the request to its list.


\end{cxxdoc}
\end{cxxfunction}
\begin{cxxfunction}
{SOAP\_INT32}
        {SOAP\_ActionInitEx}
        {(\ SOAPAction*\ action,\ HTTPManagedClient*\ httpClient,\ SOAP\_INT16\ ipType,\ URLDescriptor*\ postURL,\ URLDescriptor*\ baseURL,\ const\ SOAP\_CHAR*\ soapAction,\ SOAP\_CHAR*\ headerStr,\ SOAP\_INT32\ headerLen,\ SOAP\_CHAR*\ bodyStr,\ SOAP\_INT32\ bodyLen,\ SOAPActionCallback\ callbackFn,\ void*\ callbackData\ )}
        {}
        {2.3.2}
\cxxReturn{
\begin{tabular}[t]{lp{0.5\textwidth}}
{\tt\strut error} & code
\end{tabular}}
\begin{cxxdoc}
desc::: this will be called each time a soap action is to be performed. This will
open a http client, and send an POST, the request structure is filled and
send to be upnp client manager to add the request to its list.


\end{cxxdoc}
\end{cxxfunction}
\begin{cxxfunction}
{SOAP\_INT32}
        {SOAP\_CreateHttpSession}
        {(\ SOAPAction*\ action\ )}
        {}
        {2.3.3}
\cxxReturn{
\begin{tabular}[t]{lp{0.5\textwidth}}
{\tt\strut error} & code
\end{tabular}}
\begin{cxxdoc}
/*

\end{cxxdoc}
\end{cxxfunction}
\begin{cxxfunction}
{void}
        {SOAP\_ActionDestroy}
        {(\ SOAPAction*\ action\ )}
        {}
        {2.3.4}
\cxxReturn{
\begin{tabular}[t]{lp{0.5\textwidth}}
{\tt\strut error} & code
\end{tabular}}
\begin{cxxdoc}
close the session ::

\end{cxxdoc}
\end{cxxfunction}
\begin{cxxfunction}
{SOAP\_INT32}
        {SOAP\_ActionExecute}
        {(\ SOAPAction*\ action\ )}
        {}
        {2.3.5}
\cxxReturn{
\begin{tabular}[t]{lp{0.5\textwidth}}
{\tt\strut error} & code
\end{tabular}}
\end{cxxfunction}
\begin{cxxfunction}
{SOAP\_INT32}
        {SOAP\_ActionAddToSelectList}
        {(\ SOAPAction*\ action,\ RTP\_FD\_SET*\ readList,\ RTP\_FD\_SET*\ writeList,\ RTP\_FD\_SET*\ errList\ )}
        {}
        {2.3.6}
\cxxReturn{
\begin{tabular}[t]{lp{0.5\textwidth}}
{\tt\strut error} & code
\end{tabular}}
\begin{cxxdoc}
desc:: add to select list, and return timeout milliseconds to select for

\end{cxxdoc}
\end{cxxfunction}
\begin{cxxfunction}
{SOAP\_BOOL}
        {SOAP\_ActionProcessState}
        {(\ SOAPAction*\ action,\ RTP\_FD\_SET*\ readList,\ RTP\_FD\_SET*\ writeList,\ RTP\_FD\_SET*\ errList\ )}
        {}
        {2.3.7}
\cxxReturn{
\begin{tabular}[t]{lp{0.5\textwidth}}
{\tt\strut error} & code
\end{tabular}}
\end{cxxfunction}
\begin{cxxfunction}
{SOAP\_INT32}
        {SOAP\_SendActionRequest}
        {(\ SOAPAction*\ action\ )}
        {}
        {2.3.8}
\cxxReturn{
\begin{tabular}[t]{lp{0.5\textwidth}}
{\tt\strut error} & code
\end{tabular}}
\begin{cxxdoc}
/*

\end{cxxdoc}
\end{cxxfunction}
\begin{cxxfunction}
{SOAP\_INT32}
        {SOAP\_ReadActionResponse}
        {(\ SOAPAction*\ action\ )}
        {}
        {2.3.9}
\cxxReturn{
\begin{tabular}[t]{lp{0.5\textwidth}}
{\tt\strut error} & code
\end{tabular}}
\begin{cxxdoc}
/*

\end{cxxdoc}
\end{cxxfunction}
\begin{cxxfunction}
{SOAPActionState}
        {\_SOAP\_ParseActionRespose}
        {(\ HTTPResponseInfo*\ info,\ HTTPManagedClientSession*\ session,\ SOAPActionResponse*\ response\ )}
        {}
        {2.3.10}
\cxxReturn{
\begin{tabular}[t]{lp{0.5\textwidth}}
{\tt\strut error} & code
\end{tabular}}
\end{cxxfunction}
\end{cxxentry}
\begin{cxxentry}
{}
        {Eventing/GENA}
        {}
        {}
        {2.4}
\begin{cxxnames}
\cxxitem{GENA\_INT32}
        {GENA\_ClientInit}
        {(\ GENAClientContext*\ ctx,\ GENAClientCallback\ callback,\ void*\ cookie\ )}
        {}
        {2.4.1}
\cxxitem{void}
        {GENA\_ClientDestroy}
        {(\ GENAClientContext*\ ctx\ )}
        {}
        {2.4.2}
\cxxitem{GENA\_INT32}
        {GENA\_ClientProcessEvent}
        {(\ GENAClientContext*\ ctx,\ HTTPServerRequestContext*\ srv,\ HTTPSession*\ session,\ HTTPRequest*\ request,\ RTP\_NET\_ADDR*\ notifierAddr\ )}
        {}
        {2.4.3}
\cxxitem{GENA\_INT32}
        {GENA\_SendEventResponse}
        {(\ GENAClientEvent*\ genaEvent,\ HTTPServerRequestContext*\ srv,\ HTTPSession*\ session,\ HTTPRequest*\ request\ )}
        {}
        {2.4.4}
\cxxitem{GENA\_INT32}
        {GENA\_SubscribeRequestInit}
        {(\ GENAClientRequest*\ request,\ HTTPManagedClient*\ httpClient,\ GENA\_INT16\ ipType,\ GENA\_CHAR*\ relUrlStr,\ URLDescriptor*\ baseURL,\ GENA\_CHAR*\ callbackUrl,\ GENA\_INT32\ timeoutSec,\ GENAClientRequestCallback\ callbackFn,\ void*\ callbackData\ )}
        {}
        {2.4.5}
\cxxitem{GENA\_INT32}
        {GENA\_RenewRequestInit}
        {(\ GENAClientRequest*\ request,\ HTTPManagedClient*\ httpClient,\ GENA\_INT16\ ipType,\ GENA\_CHAR*\ serverUrl,\ URLDescriptor*\ baseURL,\ GENA\_CHAR*\ sid,\ GENA\_INT32\ timeoutSec,\ GENAClientRequestCallback\ callbackFn,\ void*\ callbackData\ )}
        {}
        {2.4.6}
\cxxitem{GENA\_INT32}
        {GENA\_UnsubscribeRequestInit}
        {(\ GENAClientRequest*\ request,\ HTTPManagedClient*\ httpClient,\ GENA\_INT16\ ipType,\ GENA\_CHAR*\ serverUrl,\ URLDescriptor*\ baseURL,\ GENA\_CHAR*\ sid,\ GENAClientRequestCallback\ callbackFn,\ void*\ callbackData\ )}
        {}
        {2.4.7}
\cxxitem{GENA\_INT32}
        {GENA\_CreateHttpSession}
        {(\ GENAClientRequest*\ request,\ HTTPManagedClient*\ httpClient,\ GENA\_CHAR*\ relUrlStr,\ URLDescriptor*\ baseURL,\ GENA\_INT16\ ipType\ )}
        {}
        {2.4.8}
\cxxitem{void}
        {GENA\_ClientRequestDestroy}
        {(\ GENAClientRequest*\ request\ )}
        {}
        {2.4.9}
\cxxitem{GENA\_INT32}
        {GENA\_ClientRequestExecute}
        {(\ GENAClientRequest*\ request\ )}
        {}
        {2.4.10}
\cxxitem{GENA\_INT32}
        {GENA\_ClientRequestAddToSelectList}
        {(\ GENAClientRequest*\ request,\ RTP\_FD\_SET*\ readList,\ RTP\_FD\_SET*\ writeList,\ RTP\_FD\_SET*\ errList\ )}
        {}
        {2.4.11}
\cxxitem{GENA\_BOOL}
        {GENA\_ClientRequestProcessState}
        {(\ GENAClientRequest*\ request,\ RTP\_FD\_SET*\ readList,\ RTP\_FD\_SET*\ writeList,\ RTP\_FD\_SET*\ errList\ )}
        {}
        {2.4.12}
\cxxitem{GENA\_INT32}
        {GENA\_SendRequest}
        {(\ GENAClientRequest*\ request\ )}
        {}
        {2.4.13}
\cxxitem{GENA\_INT32}
        {GENA\_ReadResponse}
        {(\ GENAClientRequest*\ request\ )}
        {}
        {2.4.14}
\cxxitem{GENA\_INT32}
        {\_GENA\_ReadHeader}
        {(\ void*\ userData,\ HTTPSession*\ session,\ HTTPHeaderType\ type,\ const\ GENA\_CHAR*\ name,\ const\ GENA\_CHAR*\ value)}
        {}
        {2.4.15}
\end{cxxnames}
\begin{cxxdoc}
These functions are \<b\>not\</b\> at the API level and should not be called from
code outside the UPnP library. 
\end{cxxdoc}
\begin{cxxfunction}
{GENA\_INT32}
        {GENA\_ClientInit}
        {(\ GENAClientContext*\ ctx,\ GENAClientCallback\ callback,\ void*\ cookie\ )}
        {}
        {2.4.1}
\cxxReturn{
\begin{tabular}[t]{lp{0.5\textwidth}}
{\tt\strut error} & code
\end{tabular}}
\end{cxxfunction}
\begin{cxxfunction}
{void}
        {GENA\_ClientDestroy}
        {(\ GENAClientContext*\ ctx\ )}
        {}
        {2.4.2}
\cxxReturn{
\begin{tabular}[t]{lp{0.5\textwidth}}
{\tt\strut error} & code
\end{tabular}}
\end{cxxfunction}
\begin{cxxfunction}
{GENA\_INT32}
        {GENA\_ClientProcessEvent}
        {(\ GENAClientContext*\ ctx,\ HTTPServerRequestContext*\ srv,\ HTTPSession*\ session,\ HTTPRequest*\ request,\ RTP\_NET\_ADDR*\ notifierAddr\ )}
        {}
        {2.4.3}
\cxxReturn{
\begin{tabular}[t]{lp{0.5\textwidth}}
{\tt\strut error} & code
\end{tabular}}
\end{cxxfunction}
\begin{cxxfunction}
{GENA\_INT32}
        {GENA\_SendEventResponse}
        {(\ GENAClientEvent*\ genaEvent,\ HTTPServerRequestContext*\ srv,\ HTTPSession*\ session,\ HTTPRequest*\ request\ )}
        {}
        {2.4.4}
\cxxReturn{
\begin{tabular}[t]{lp{0.5\textwidth}}
{\tt\strut error} & code
\end{tabular}}
\end{cxxfunction}
\begin{cxxfunction}
{GENA\_INT32}
        {GENA\_SubscribeRequestInit}
        {(\ GENAClientRequest*\ request,\ HTTPManagedClient*\ httpClient,\ GENA\_INT16\ ipType,\ GENA\_CHAR*\ relUrlStr,\ URLDescriptor*\ baseURL,\ GENA\_CHAR*\ callbackUrl,\ GENA\_INT32\ timeoutSec,\ GENAClientRequestCallback\ callbackFn,\ void*\ callbackData\ )}
        {}
        {2.4.5}
\cxxReturn{
\begin{tabular}[t]{lp{0.5\textwidth}}
{\tt\strut error} & code
\end{tabular}}
\begin{cxxdoc}
desc::: this will be called each time a GENA request is to be performed. This will
open a http client, and send an POST, the request structure is filled and
send to be upnp client manager to add the request to its list.


\end{cxxdoc}
\end{cxxfunction}
\begin{cxxfunction}
{GENA\_INT32}
        {GENA\_RenewRequestInit}
        {(\ GENAClientRequest*\ request,\ HTTPManagedClient*\ httpClient,\ GENA\_INT16\ ipType,\ GENA\_CHAR*\ serverUrl,\ URLDescriptor*\ baseURL,\ GENA\_CHAR*\ sid,\ GENA\_INT32\ timeoutSec,\ GENAClientRequestCallback\ callbackFn,\ void*\ callbackData\ )}
        {}
        {2.4.6}
\cxxReturn{
\begin{tabular}[t]{lp{0.5\textwidth}}
{\tt\strut error} & code
\end{tabular}}
\begin{cxxdoc}
desc::: this will be called each time a GENA request is to be performed. This will
open a http client, and send an POST, the request structure is filled and
send to be upnp client manager to add the request to its list.


\end{cxxdoc}
\end{cxxfunction}
\begin{cxxfunction}
{GENA\_INT32}
        {GENA\_UnsubscribeRequestInit}
        {(\ GENAClientRequest*\ request,\ HTTPManagedClient*\ httpClient,\ GENA\_INT16\ ipType,\ GENA\_CHAR*\ serverUrl,\ URLDescriptor*\ baseURL,\ GENA\_CHAR*\ sid,\ GENAClientRequestCallback\ callbackFn,\ void*\ callbackData\ )}
        {}
        {2.4.7}
\cxxReturn{
\begin{tabular}[t]{lp{0.5\textwidth}}
{\tt\strut error} & code
\end{tabular}}
\begin{cxxdoc}
desc::: this will be called each time a GENA request is to be performed. This will
open a http client, and send an POST, the request structure is filled and
send to be upnp client manager to add the request to its list.


\end{cxxdoc}
\end{cxxfunction}
\begin{cxxfunction}
{GENA\_INT32}
        {GENA\_CreateHttpSession}
        {(\ GENAClientRequest*\ request,\ HTTPManagedClient*\ httpClient,\ GENA\_CHAR*\ relUrlStr,\ URLDescriptor*\ baseURL,\ GENA\_INT16\ ipType\ )}
        {}
        {2.4.8}
\cxxReturn{
\begin{tabular}[t]{lp{0.5\textwidth}}
{\tt\strut error} & code
\end{tabular}}
\begin{cxxdoc}
/*

\end{cxxdoc}
\end{cxxfunction}
\begin{cxxfunction}
{void}
        {GENA\_ClientRequestDestroy}
        {(\ GENAClientRequest*\ request\ )}
        {}
        {2.4.9}
\cxxReturn{
\begin{tabular}[t]{lp{0.5\textwidth}}
{\tt\strut error} & code
\end{tabular}}
\end{cxxfunction}
\begin{cxxfunction}
{GENA\_INT32}
        {GENA\_ClientRequestExecute}
        {(\ GENAClientRequest*\ request\ )}
        {}
        {2.4.10}
\cxxReturn{
\begin{tabular}[t]{lp{0.5\textwidth}}
{\tt\strut error} & code
\end{tabular}}
\end{cxxfunction}
\begin{cxxfunction}
{GENA\_INT32}
        {GENA\_ClientRequestAddToSelectList}
        {(\ GENAClientRequest*\ request,\ RTP\_FD\_SET*\ readList,\ RTP\_FD\_SET*\ writeList,\ RTP\_FD\_SET*\ errList\ )}
        {}
        {2.4.11}
\cxxReturn{
\begin{tabular}[t]{lp{0.5\textwidth}}
{\tt\strut error} & code
\end{tabular}}
\begin{cxxdoc}
desc:: add to select list, and return timeout milliseconds to select for

\end{cxxdoc}
\end{cxxfunction}
\begin{cxxfunction}
{GENA\_BOOL}
        {GENA\_ClientRequestProcessState}
        {(\ GENAClientRequest*\ request,\ RTP\_FD\_SET*\ readList,\ RTP\_FD\_SET*\ writeList,\ RTP\_FD\_SET*\ errList\ )}
        {}
        {2.4.12}
\cxxReturn{
\begin{tabular}[t]{lp{0.5\textwidth}}
{\tt\strut error} & code
\end{tabular}}
\end{cxxfunction}
\begin{cxxfunction}
{GENA\_INT32}
        {GENA\_SendRequest}
        {(\ GENAClientRequest*\ request\ )}
        {}
        {2.4.13}
\cxxReturn{
\begin{tabular}[t]{lp{0.5\textwidth}}
{\tt\strut error} & code
\end{tabular}}
\begin{cxxdoc}
/*

\end{cxxdoc}
\end{cxxfunction}
\begin{cxxfunction}
{GENA\_INT32}
        {GENA\_ReadResponse}
        {(\ GENAClientRequest*\ request\ )}
        {}
        {2.4.14}
\cxxReturn{
\begin{tabular}[t]{lp{0.5\textwidth}}
{\tt\strut error} & codeIMPORTANT - response.sid is allocated memory in read header callback, it needs to be freed
\end{tabular}}
\begin{cxxdoc}
/*

\end{cxxdoc}
\end{cxxfunction}
\begin{cxxfunction}
{GENA\_INT32}
        {\_GENA\_ReadHeader}
        {(\ void*\ userData,\ HTTPSession*\ session,\ HTTPHeaderType\ type,\ const\ GENA\_CHAR*\ name,\ const\ GENA\_CHAR*\ value)}
        {}
        {2.4.15}
\cxxReturn{
\begin{tabular}[t]{lp{0.5\textwidth}}
{\tt\strut error} & code
\end{tabular}}
\begin{cxxdoc}
/*

\end{cxxdoc}
\end{cxxfunction}
\end{cxxentry}
\end{cxxentry}
\end{document}
